\documentclass[12pt, conceptpaper]{article}
\usepackage{amsmath}
\begin{document}
    \title{Investigation of the solution of least squares problems using the QR factorization}
    \author{KIKOMEKO MUSA 15/U/6675/PS \\}
    \date{16/04/2017}
    \maketitle
\section{METHODOLOGY}
\paragraph{Among the methods includes the following:the normal equationa.}
\paragraph{Given data $((x1; y1).......(xN; yN))$, we may define the error associated to saying y = ax + b.}
\paragraph{This is just N times the variance of the data set and It makes no difference whether or not we study the variance or N times the variance as our error, and note that the error is a function of two variables.}
\paragraph{The goal is to find values of a and b that minimize the error.We will describe how to factor a general m × n matrix A, with m ≥ n,A = QˆR.}
	\begin{equation}
a^2y-(ax+b)=1/N\sum_{N}^{n=1}){(yn ¡ (axn + b^2))}.
	\end{equation}



\cleardoublepage
\bibliographystyle{IEEEtran}
\section{REFERENCES}\label{sec:intro}
{American Congress on Surveying and Mapping,
	author = {American Congress on Surveying and Mapping},
	title = {Issue of surveying and Land Information System},
	date = {June, 2001},
\end{document}